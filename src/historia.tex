\frame
{
\frametitle{Historia}
\begin{columns}
\column{0.5 \textwidth}
\begin{itemize}
\item La UTFSM ha tenido un activo rol impulsando la \textbf{computación de
alto desempeño} (High Performance Computing, HPC) en Chile participando en
proyectos
de investigación aplicados con universidades y centros de investigación
nacionales e internacionales. 
	\begin{itemize}
		\item EELA, EELA-2
		\item SCAT
		\item EPIKH, GISELA
		\item NLHPC
	\end{itemize}
\end{itemize}
\column{0.5\textwidth}
\pgfuseimage{utfsm}
\end{columns}
}


\frame
{
\frametitle{Historia}
\begin{columns}
\column{0.6\textwidth}
\begin{itemize}
\item El interés por la HPC se formaliza en el año 2008 con la creación del
Centro de Innovación Tecnológica en Computación de Alto Desempeño (CTI-HPC,
\texttt{www.hpc.usm.cl}).

\end{itemize}
\column{0.4\textwidth}
\pgfuseimage{bari}
\end{columns}
}



%La UTFSM ha tenido un activo rol impulsando la computación de alto desempeño (High Performance Computing, HPC) en Chile participando en proyectos de investigación aplicados con universidades y centros de investigación nacionales e internacionales. El interés por la HPC en la UTFSM se formaliza con la creación en el año 2008 del Centro de Innovación Tecnológica en Computación de Alto Desempeño (CTI-HPC, www.hpc.usm.cl), unidad actual enfocada en promover y contribuir en esta área.


\frame
{
\frametitle{CTI-HPC}
        \begin{itemize}
	    \item El CTI-HPC es la unidad actual enfocada en promover y
contribuir en la computación de alto desempeño.
	    \item Áreas de Trabajo	
		\begin{itemize}
		\item Finanzas computacionales
		\item Procesamiento de Imágenes
		\item Bioinformática
		\item Cuda Teaching Center
		\item Capacitación y entrenamiento en HPC
		\end{itemize}
        \end{itemize}

